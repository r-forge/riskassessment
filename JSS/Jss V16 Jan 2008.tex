%% LyX 1.6.1 created this file.  For more info, see http://www.lyx.org/.
%% Do not edit unless you really know what you are doing.
\documentclass[article]{jss}
\usepackage[T1]{fontenc}
\usepackage[latin9]{inputenc}
\usepackage[authoryear]{natbib}

\makeatletter
%%%%%%%%%%%%%%%%%%%%%%%%%%%%%% User specified LaTeX commands.
%the following commands are used only for articles and codesnippets

\author{R�gis Pouillot\\CFSAN \And 
        Marie Laure Delignette-Muller\\Universit� de Lyon \AND
        Jean-Baptiste Denis\\INRA}
\title{\pkg{fitdistrplus} and \pkg{mc2d}, Two Packages for Risk Assessment in \proglang{R}}

% the same as above, without any formatting
\Plainauthor{R�gis Pouillot, Marie Laure Delignette Muller, Jean-Baptiste Denis}
\Plaintitle{fitdistrplus and mc2d, two packages for risk assessment in R} 
%if necessary, provide a short title
\Shorttitle{Package fitdistrplus and mc2d in R} 

\Abstract{This is the abstract of the article.}
%at least one keyword is needed
\Keywords{keyword 1, keyword 2, \pkg{fitdistrplus}, \pkg{mc2d}, \proglang{R}}
%the same as above, without any formatting
\Plainkeywords{keyword 1, keyword 2, foo, R} 


%The address of at least one author should be given in the following format
\Address{
  R�gis Pouillot\\
  Center for Food Safety and Applied Nutrition\\
  FDA/HHS\\
  College Park, USA\\
  E-mail: \email{rpouillot@yahoo.fr}\\
  URL: \url{http://www.lyx.org}
}

%% need no \usepackage{Sweave.sty}

%you can add a telephone and fax number before the e-mail in the format
%Telephone: +12/3/4567-89
%Fax: +12/3/4567-89

%if you use Sweave,  include the following line (with % symbols):
%% need no \usepackage{Sweave.sty}

\makeatother

\begin{document}


\section[Introduction]{Introduction}

Ceci est un essai de bibliographie \citealt{CULLEN-FREY-1999}.


\subsection{Variability and uncertainty in the risk assessment framework }

According to international recommendations, a QRA should reflect the
{}``variability'' in the risk and calculate the {}``uncertainty''
associated with the risk estimate. The variability represents temporal,
geographical and/or individual heterogeneity of the risk for a given
population. The {}``uncertainty'' is understood as stemming from
a lack of perfect knowledge about the QRA model structure and associated
parameters%
\footnote{In the engineering risk community, these concepts are refered as \textquotedbl{}aleatoric
uncertainty\textquotedbl{} for \textquotedbl{}variability\textquotedbl{}
and \textquotedbl{}epistemic uncertainty\textquotedbl{} for \textquotedbl{}uncertainty\textquotedbl{}.%
}.

In order to estimate the natural {}``variability'' of the risk,
a Monte-Carlo simulation approach may be useful: the empirical distribution
of the risk within the population may be estimated from the mathematical
combination of distributions reflecting the variability of parameters
across the population.

A two-dimensional (or second-order) Monte-Carlo simulation was proposed
to estimate the {}``uncertainty'' in the risk estimates stemming
from parameter uncertainty. A two-dimensional Monte-Carlo simulation
is a Monte-Carlo simulation where the distributions reflecting \textquotedbl{}variability\textquotedbl{}
and the distributions representing \textquotedbl{}uncertainty\textquotedbl{}
are sampled separately in the simulation, so that \textquotedbl{}variability\textquotedbl{}
and \textquotedbl{}uncertainty\textquotedbl{} in the output may be
estinated separately. 


\subsection{The need for a package for risk assessment}

Integrated 


\subsection{An example}

\emph{Escherichia coli} O157:H7 in ground beef.


\section[Using fitdistrplus]{Using \pkg{fitdistrplus}}

To evaluate the intake distribution


\section[Using mc2d]{Using \pkg{mc2d}}

To derive the risk


\section{Conclusions}


\section{To be discarded}

\proglang{R} for the names of programming languages, \pkg{foo}
for software packages, and \code{some code} or \pkg{some code}
for code; $\E$  for expectations, $\VAR$  for variances, $\COV$
 for covariances, and $\Prob$  for probabilities.

Writing several lines of code: 

\begin{CodeChunk}
\begin{Code}
first line of code
second line of code
third line of code
\end{Code}
\end{CodeChunk}

Alternatively, you can distinguish between input and output code:

\begin{CodeChunk}
\begin{CodeInput}
R> library(mc2d)
R> ndvar(1001)
\end{CodeInput}
\begin{CodeOutput}
[1] 1001 
\end{CodeOutput}
\begin{CodeInput}
ndunc(1001)
\end{CodeInput}
\begin{CodeOutput}
[1] 1001 
\end{CodeOutput}
\end{CodeChunk}

Or

\begin{CodeChunk}
\begin{CodeInput}
library(mc2d)
ndvar(1001)
\end{CodeInput}
\begin{CodeOutput}
[1] 1001 
\end{CodeOutput}
\begin{CodeInput}
ndunc(1001)
\end{CodeInput}
\begin{CodeOutput}
[1] 1001 
\end{CodeOutput}
\end{CodeChunk}


\section*{Acknowledgments}

Here you can write some acknowledgments.

\bibliographystyle{jss}
\bibliography{BibJss}


\end{document}
